\documentclass{report}

\title{PDL Manual}
\author{}
\date{}

\usepackage{titlesec}
\usepackage{natbib}
\usepackage{graphicx}
\usepackage{listings}
\lstset{
    basicstyle=\ttfamily,
    frame=single,
    escapeinside=``
}

\begin{document}

\titleformat{\chapter}[hang]{\centering\LARGE\bfseries}{Chapter~\thechapter}{1em}{}

\maketitle

\tableofcontents

\thispagestyle{empty}

\setcounter{page}{1}

\chapter{Introduction}

Problem Description Language (PDL) is a modeling language for specifying constraint optimization problems. When using PDL, users only need to list all constraints between the variables without indicating how to solve the problem -- the PDL runtime system will automatically design a feasible solving algorithm and generate the corresponding code.

PDL supports many common-used operators and functions. PDL lets users describe models in a way that is close to a mathematical formulation of the problem using familiar notations.

Compared with programming with traditional programming languages, programming with PDL has a lot of advantages:
\begin{itemize}
    \item Users can solve the problems without any programming abilities, only some basic mathematical knowledge is required;
    \item PDL models are more readable than programs, which makes it easier to comprehense and debug;
    \item Users can focus on modeling, and save plenty of time in thinking of how to solve the problem;
    \item Bugs introduced by human errors can be avoided while using an automatic generation tool;
    \item Users can apply advanced optimizations based on the generated programs.
\end{itemize}

\chapter{PDL Examples}

This chapter demonstrates PDL models of two problems. \textbf{White space can be freely inserted between tokens to improve readability.} Detailed explanations of usage will be provided in Chapter 3.

\section{Cuboid Problem}

Given the volume $V$, find a cuboid with the smallest surface area such that (1) the volume is V and (2) the lengths of all edges are integers.

\begin{lstlisting}
#input
	V of int in [1,10^4];

#required
	L of int in [1,?];
	W of int in [1,?];
	H of int in [1,?];
	V = L * W * H;

#objective
	minimize (2 * (L * W + L * H + W * H));
\end{lstlisting}

\section{0/1 Knapsack Problem}

There are $N$ items, each has value and weight. Select a set of items to put in the knapsack, such that the total weight is not greater than the capacity and the total value is maximum.

\begin{lstlisting}
#input
	N of int in [1,100];
	capacity of int in [1,1000];
	profits of (int in [1,1000])[1~N];
	weights of (int in [1,1000])[1~N];
	
#required
	knapsack of (int in [1,N]){};
	summation
	    [weights[i] : forall i (i in knapsack)] 
	    <= capacity;
	    
#objective
	maximize summation
	    [profits[i] : forall i (i in knapsack)];
\end{lstlisting}

\chapter{PDL Tutorial}

\section{Basic Framework}

A PDL model consists of three segments: \texttt{\#input}, \texttt{\#required}, and \texttt{\#objective}. Each statement should be ended with a semicolon (``\texttt{;}''). We will take 0/1 knapsack problem as an example to describe the basic framework of a PDL model in detail.

\subsection{\texttt{\#input} Section}

This section declares all input parameters and their types as well as value ranges.

In the example, the \texttt{\#input} section lists four input parameters of this problem: two variables \texttt{N} and \texttt{capacity} of integer type, and two variables \texttt{profits} and \texttt{weights} of integer array type. Meanwhile, the value ranges as well as index ranges are also declared.

\textbf{Note: These inputs will be read in the same order by the generated program, so please make sure that the declaring order is correct.}

\subsection{\texttt{\#required} Section}

This section contains several statements. Each statement can be:
\begin{itemize}
    \item a variable declaration, which is similar to the one in \texttt{\#input} section, or
    \item a boolean expression indicating a mathematical relation between inputs and variables.
\end{itemize}

In the example, \texttt{\#required} section first define a variable \texttt{knapsack} of set type---the indexes of items we are going to pick. Then follows a constraint that should be satisfied---the total weight of the selected items should not exceeding \texttt{capacity}.

\subsection{\texttt{\#objective} Section}

This section declares zero or one objective function. If no objective function is defined, the resulting program will stop search after finding one solution; otherwise, it will search for the optimal solution that produces either the maximized or minimized value.

In the example, the objective function is to maximize the total profits of the selected items.

\section{Variable Declaration}

In PDL, input parameters or variables are declared by:

\begin{lstlisting}
    <identifier> of <type-and-range>
\end{lstlisting}

\subsection{Identifier}

An identifier of a user-defined item (input, variable, function ...) should be a sequence of letters (\texttt{a-z} or \texttt{A-Z}, case-sensitive), numbers (\texttt{0-9}), and underscores (\texttt{\_}). The first character should be a letter. Some reserved keywords cannot be used as identifiers, including:
\begin{verbatim}
    int, real, bool, char, function, of, in, true, false,
    and, or, not, xor, mod, if, else, forall, exists, 
    summation, product, count, min, max, minimize, maximize
\end{verbatim}
\textbf{Some reserved keywords in C/C++ cannot be used either}, such as \texttt{return}, \texttt{main}, etc.

\subsection{Type and Range}

PDL supports four primitive types and three composite types.

Primitive types are: integer type \texttt{int}, real type \texttt{real}, boolean type \texttt{bool}, and character type \texttt{char}. To limit range of a primitive type \texttt{T} between \texttt{lb} and \texttt{ub}, use \texttt{T in [lb,ub]}. Specially, use \texttt{?} if a bound of the range is unspecified.

Composite types are:
\begin{itemize}
    \item Array type: an array of \texttt{T} type variables is represented as \texttt{T[]}, for example, \texttt{int[]} is an array of integers, \texttt{bool[][]} is a two-dimensional array of booleans. To limit the range of index between \texttt{li} and \texttt{ui}, use \texttt{T[li$\scriptsize{\sim}$ui]}. Similarly, use \texttt{?} if a bound is unspecified.
    \item Set type: a set of \texttt{T} type variables is represented as \texttt{T\{\}}.
    \item Tuple type: a tuple can include variables of different types, which is represented as \texttt{(T1,T2,...)}.
\end{itemize}

\subsection{Examples of Variable Declaration}

\begin{lstlisting}
// an integer variable
i of int
// an integer variable with value range
n of int in [1,10^5]
// an integer array
a of int[]
// an integer array with index range
b of int[1`$\scriptsize{\sim}$`2*n]
// an integer array with both range and index ranges
c of (int in [1,10^6])[1`$\scriptsize{\sim}$`100]
// a multi-dimensional array with index ranges
d of (int in [1,100])[0`$\scriptsize{\sim}$`?][][-5`$\scriptsize{\sim}$`5]
// an integer set
s of int{}
// an integer set with value range
t of (int in [1,n]){}
// a tuple consists of elements of different types
p of (int, bool[1`$\scriptsize{\sim}$`10], int in [1,?])
// a set of tuples
q of (int, bool){}
\end{lstlisting}

\section{Primitive Operator}

PDL supports many common-used mathematical operators on primitive types,  listed from greatest to smallest priority: 
\begin{enumerate}
    \item parenthesis \texttt{()}
    \item power \texttt{\^{}}
    \item multiply \texttt{*}, divide \texttt{/}, modulus \texttt{mod}
    \item add \texttt{+}, subtract \texttt{-}
    \item type check \texttt{of}, range check \texttt{in}
    \item \texttt{not}, \texttt{and}
    \item \texttt{or}, \texttt{xor}
    \item comparisons \texttt{=}, \texttt{!=}, \texttt{>}, \texttt{<}, \texttt{>=}, \texttt{<=}
\end{enumerate}

\section{Composite Operator}

PDL also supports some built-in operator for composite types:
\begin{itemize}
    \item Element selections: \texttt{[]}, the priority of which is second only to \texttt{()}. For instance, \texttt{a[5]}, \texttt{a[i+1]}, where \texttt{a} is an array or tuple. Note that the index of elements in a tuple starts at 1.
    \item Set operators: intersect \texttt{*}, union \texttt{+}, minus \texttt{-}. Each set operator has the same priority as the primitive operator sharing the same notation.
    \item Set comparisons, similar to primitive comparisons.
    \item Belong to: \texttt{in}, the priority of which equals to the comparisons'. An expression \texttt{e in C} returns a boolean value, indicating whether there exists an element \texttt{e} in the variable \texttt{C} of a composite type.
    \item Aggregate operators: \texttt{summation}, \texttt{product}, \texttt{count}, \texttt{max}, \texttt{min}. These operators can be applied to an array or set. For instance, \texttt{summation a} calculates the sum of all elements in array \texttt{a}, \texttt{max s} obtains the maximum element in set \texttt{s}.
\end{itemize}

\section{Quantifier}

PDL supports two quantifiers $\exists$ and $\forall$ in first-order logic, represented by \texttt{exists} and \texttt{forall}.

\subsection{Existential Quantifier}

An existential quantifier can be used as
\begin{lstlisting}
exists <v1, v2, ...> (<bool-exp>)
\end{lstlisting}
The expression returns a boolean value, indicating whether there exists an assignment to all variables in the list that satisfies the following boolean expression.

Example: checking whether there exists an inversion in the array \texttt{a}.
\begin{lstlisting}
exists i,j ((i<j) and (a[i]>a[j]))
\end{lstlisting}

\subsection{Universal Quantifier}

In PDL, a universal quantifier can be used for two different purposes.

\paragraph{Generating an array}:
\begin{lstlisting}
[<exp> : forall <v1, v2, ...> (<bool-exp>)]
\end{lstlisting}
This statement enumerates each value combination of variables that satisfies the boolean expression, generates an element by the front expression, and puts it into an array.

Example: select each element from array \texttt{a} whose index belongs to set \texttt{s}.
\begin{lstlisting}
[ a[i] : forall i (i in s) ]
\end{lstlisting}

\paragraph{Generating statements}:
\begin{lstlisting}
<bool-exp> : forall <v1, v2, ...> (<bool-exp>)
\end{lstlisting}
This statement enumerates each value combination of variables that satisfies the latter boolean expression, generates a sequence of constraints by the first boolean expression.

Example: require the elements in array \texttt{a} are in increasing order.
\begin{lstlisting}
a[i] < a[i+1] : forall i ( i of int in [1,N-1] )
\end{lstlisting}
\textbf{Note: the boolean expression after \texttt{forall} can be omitted if there is no extra constraint on the variables in the list.}

\section{Conditional Statement}

\begin{lstlisting}
if <bool-exp> <exp1> else <exp2>
\end{lstlisting}
\textbf{Note: \texttt{else} and the expression after can be omitted.}

\section{Function Define and Call}

In PDL, users can self-define a function to simplify the description, or achieve recursive calculations.

Functions can be defined by:
\begin{lstlisting}
<identifier> of function (<v1-def, v2-def, ...>)
    -> <type-and-range> = <exp>
\end{lstlisting}
where \texttt{identifier} is the function name, \texttt{(v1-def,v2-def,...)->type-and-range} defines the type of function arguments and the return value. The last expression defines how to calculate the return value.

Functions can be called by:
\begin{lstlisting}
<identifier>(<v1, v2,...>)
\end{lstlisting}

Example: check whether an integer \texttt{x} is an odd number.

Function define:
\begin{lstlisting}
isOdd of function (x of int) -> bool = (x mod 2 = 1)
\end{lstlisting}

Function call:
\begin{lstlisting}
isOdd(5)
\end{lstlisting}

%\bibliographystyle{plain}
%\bibliography{references}
\end{document}
